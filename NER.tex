\documentclass{report}\usepackage[]{graphicx}\usepackage[]{color}
%% maxwidth is the original width if it is less than linewidth
%% otherwise use linewidth (to make sure the graphics do not exceed the margin)
\makeatletter
\def\maxwidth{ %
  \ifdim\Gin@nat@width>\linewidth
    \linewidth
  \else
    \Gin@nat@width
  \fi
}
\makeatother

\definecolor{fgcolor}{rgb}{0.345, 0.345, 0.345}
\newcommand{\hlnum}[1]{\textcolor[rgb]{0.686,0.059,0.569}{#1}}%
\newcommand{\hlstr}[1]{\textcolor[rgb]{0.192,0.494,0.8}{#1}}%
\newcommand{\hlcom}[1]{\textcolor[rgb]{0.678,0.584,0.686}{\textit{#1}}}%
\newcommand{\hlopt}[1]{\textcolor[rgb]{0,0,0}{#1}}%
\newcommand{\hlstd}[1]{\textcolor[rgb]{0.345,0.345,0.345}{#1}}%
\newcommand{\hlkwa}[1]{\textcolor[rgb]{0.161,0.373,0.58}{\textbf{#1}}}%
\newcommand{\hlkwb}[1]{\textcolor[rgb]{0.69,0.353,0.396}{#1}}%
\newcommand{\hlkwc}[1]{\textcolor[rgb]{0.333,0.667,0.333}{#1}}%
\newcommand{\hlkwd}[1]{\textcolor[rgb]{0.737,0.353,0.396}{\textbf{#1}}}%
\let\hlipl\hlkwb

\usepackage{framed}
\makeatletter
\newenvironment{kframe}{%
 \def\at@end@of@kframe{}%
 \ifinner\ifhmode%
  \def\at@end@of@kframe{\end{minipage}}%
  \begin{minipage}{\columnwidth}%
 \fi\fi%
 \def\FrameCommand##1{\hskip\@totalleftmargin \hskip-\fboxsep
 \colorbox{shadecolor}{##1}\hskip-\fboxsep
     % There is no \\@totalrightmargin, so:
     \hskip-\linewidth \hskip-\@totalleftmargin \hskip\columnwidth}%
 \MakeFramed {\advance\hsize-\width
   \@totalleftmargin\z@ \linewidth\hsize
   \@setminipage}}%
 {\par\unskip\endMakeFramed%
 \at@end@of@kframe}
\makeatother

\definecolor{shadecolor}{rgb}{.97, .97, .97}
\definecolor{messagecolor}{rgb}{0, 0, 0}
\definecolor{warningcolor}{rgb}{1, 0, 1}
\definecolor{errorcolor}{rgb}{1, 0, 0}
\newenvironment{knitrout}{}{} % an empty environment to be redefined in TeX

\usepackage{alltt}

\usepackage{polyglossia}
  \setmainlanguage{spanish}
\usepackage{fontspec}
\usepackage{geometry}
\usepackage{parskip}
\usepackage{tcolorbox}
\usepackage{mathtools}
  \newtagform{brackets}{[}{]}
  \usetagform{brackets}
\usepackage{subcaption}
\usepackage{pgfplots}
  \pgfplotsset{compat=1.12}
  \makeatletter
    \pgfdeclareplotmark{dot}
      {%
        \fill circle [x radius=0.04, y radius=0.16];
      }%
  \makeatother
\usepackage{booktabs}
\usepackage{makeidx}
  \makeindex
\usepackage[backend=bibtex,style=authoryear,citestyle=authoryear,doi=false,isbn=false,url=false]{biblatex}
  \addbibresource{NER-bib.bib}
\usepackage[hidelinks,spanish]{hyperref}

%pgfplot función normal
\pgfmathdeclarefunction{gauss}{2}{%
  \pgfmathparse{1/(#2*sqrt(2*pi))*exp(-((x-#1)^2)/(2*#2^2))}%
}

\newcommand*{\eng}[1]{\textsl{#1}}
\newcommand*{\paq}[1]{\textbf{#1}\index{#1@\textbf{#1}}}
\newcommand*{\kw}[1]{\textbf{#1}\index{#1}}

\newcommand{\Var}{\operatorname{Var}}
\newcommand{\Cov}{\operatorname{Cov}}
\newcommand{\E}{\operatorname{E}}
\newcommand{\sd}{\operatorname{sd}}
\newcommand{\se}{\operatorname{se}}

% Eliminar numeración en títulos menores que secciones
\setcounter{secnumdepth}{2}

% Color box para R
% new tcolorbox environment
% #1: tcolorbox options
% #2: color
% #3: box title
\newtcolorbox{Rbox}[1][]
{
  title = El c\'odigo R,
  fonttitle=\sffamily\bfseries,
  colbacktitle=black!35!white,
  colframe=black!35!white,
  fontupper=\sffamily,
  colupper=black!80!white
  #1,
}

% Code chunk



\title{Notas de Econometría en R}
\author{Alvaro Carril\thanks{Estas notas son un intento de condensar lo justo y necesario para poder seguir un curso de econometría con R.
Luego de recopilar material de muchas clases y ayudantías de econometría, en general pensadas para ser aplicadas en Stata, decidí que podía intentar hacer estas clases en R y usar esta guía como material de apoyo.
Mi prioridad fue ser breve y conciso, por lo que estas notas no son un buen comienzo para aprender R (o econometría); son un complemento.
Si hay algún error o quieres hacer algún comentario, mi correo es \texttt{acarril@fen.uchile.cl}.}}
\IfFileExists{upquote.sty}{\usepackage{upquote}}{}
\begin{document}

\maketitle
\tableofcontents

\clearpage

% Introducción, prefacio:
%<<cap_intro, child='cap_intro.Rnw'>>=
%@

% Introducción, prefacio:



\chapter{Regresión lineal simple}



El modelo de regresión lineal es el caballo de batalla de la econometría aplicada.
Muchas de las preguntas que abordaremos comienzan con dos variables, digamos $x$ e $y$. Nos interesará entonces ``explicar $y$ en términos de $x$''.
Por ejemplo, $y$ podría ser la tasa de crímenes de varios barrios e $x$ el número de parques construidos en esos barrios.
O $y$ podría ser el porcentaje de los votos obtenidos por un partido y $x$ su gasto en la campaña electoral, etc.

La \autoref{fig:votos_plot} muestra datos para 173 campañas entre dos candidatos en EE.UU, enfocándose en el ``candidato A'' (un nombre del que Kafka sin duda estaría orgulloso).
El eje $x$ corresponde al porcentaje gastado por el candidato A (\verb|shareA|), mientras que el eje $y$ indica el porcentaje de los votos obtenidos por dicho candidato (\verb|voteA|).
El código para crear el gráfico se muestra a continuación:
\begin{knitrout}
\definecolor{shadecolor}{rgb}{0.969, 0.969, 0.969}\color{fgcolor}\begin{kframe}
\begin{alltt}
\hlkwd{library}\hlstd{(foreign)}
\hlkwd{library}\hlstd{(ggplot2)}
\hlstd{votos} \hlkwb{<-} \hlkwd{read.dta}\hlstd{(}\hlstr{"http://fmwww.bc.edu/ec-p/data/wooldridge/vote1.dta"}\hlstd{)}
\hlkwd{ggplot}\hlstd{(votos,} \hlkwd{aes}\hlstd{(shareA, voteA))} \hlopt{+} \hlkwd{geom_point}\hlstd{()}
\end{alltt}
\end{kframe}\begin{figure}

{\centering \includegraphics[width=4in,height=3in]{figs/votos_plot-1} 

}

\caption[Relación entre porcentaje de gasto (\texttt{shareA}) y porcentaje de votos obtenidos (\texttt{voteA})]{Relación entre porcentaje de gasto (\texttt{shareA}) y porcentaje de votos obtenidos (\texttt{voteA})}\label{fig:votos_plot}
\end{figure}


\end{knitrout}
\begin{Rbox}
El comando \verb|library()| permite cargar paquetes de R, los que contienen más funciones y datos. \paq{foreign} es un paquete que nos permite usar la función \verb|read.dta()| para leer una base de datos de Stata, y \paq{ggplot2} nos permite usar la funciones \verb|ggplot()| y \verb|geom_point()| para crear un gráfico de puntos.
\end{Rbox}

Al intentar modelar la relación entre $x$ e $y$ surgen una serie de preguntas. Quizás la más inmediata es: ``¿Cuál es la relación funcional entre ambas variables?''
Nuestro primer supuesto será que las variables se relacionan de manera lineal, es decir,
\begin{equation}
y = ax + b.
\label{eq:ec_lin}
\end{equation}
Notemos que esta ecuación es idéntica a la clásica función lineal de colegio, y su interpretación geométrica es la misma: $a$ es un término que representa la pendiente de la recta y $b$ es el intercepto.
Este supuesto de relación lineal parece cumplirse bastante bien para nuestro ejemplo de la \autoref{fig:votos_plot}: no sabemos qué valores deberían tomar $a$ y $b$, pero definitivamente pareciera existir alguna combinación de valores de $a$ y $b$ tal que una línea se ajuste ``bien'' a los datos.

Sin embargo, hasta ahora nuestro modelo expresado en \eqref{eq:ec_lin} asume que existe una relación lineal \emph{perfecta} entre $x$ e $y$, lo que es altamente restrictivo. 
En términos de la \autoref{fig:votos_plot}, es como suponer que podríamos trazar una recta que pasara por \emph{todos} los puntos, lo que evidentemente no se puede.
Entonces surge naturalmente la pregunta de si es posible permitir que otros factores ---distintos de $x$--- afecten en cierta medida a $y$, de forma que nuestro modelo tenga un margen de error.
Esto se logra agregando un \kw{término de error}, denotado por $\mu$, al modelo lineal.

Adicionalmente, por un simple tema de convención llamaremos $\beta_0$ al parámetro del intercepto y $\beta_1$ al parámetro de la pendiente.
Entonces podemos reescribir nuestro modelo, que ahora es
\begin{equation}
y = \beta_0 + \beta_1x + \mu.
\label{eq:modelo_lineal_simple_poblacional}
\end{equation}
Esta ecuación define el \kw{modelo de regresión lineal simple}. Entendamos bien qué quiere decir cada palabra de este título:
\begin{itemize}
\item \textbf{modelo}, porque es un intento de simplificar la realidad para enfocarnos en algunos aspectos que nos interesan
\item \textbf{de}, una preposición en la gramática del español
\item \textbf{regresión} es el término usado originalmente por \textcite{galton_regression_1886}, aunque conceptualmente no guarda mucha relación con lo que hacemos ahora\footnote{\textcite{stigler_history_1986} cuenta una historia más detallada al respecto.} (pero el nombre es \eng{cool})
\item \textbf{lineal}, porque estamos asumiendo que los aspectos de la realidad que nos interesan ---las variables $x$ e $y$--- se relacionan de manera lineal
\item \textbf{simple}, porque estamos restringiendo el análisis a solamente dos variables (por el momento).
\end{itemize}

Muchos libros tienen nombres distintos para $x$ e $y$. Yo uso la convención (bastante establecida) de referirme a $y$ como la \kw{variable dependiente} del modelo, y a $x$ como la \kw{variable independiente} del modelo. Estas son las variables que ya tenemos, y que intentaremos relacionar.
Ya mencionamos que $\mu$ es el llamado \kw{término de error}, que captura factores que afectan a $y$ que no hemos incluido en el modelo. Ojo que $\mu$ no es igual a $u$, y se pronuncia <<mu>>. Esto es sólo para enredar, claro.

En conjunto nos referiremos a $\beta_0$ y $\beta_1$ como los \kw{parámetros} del modelo. Estos son simples números, pero como no los tenemos, intentaremos \emph{estimarlos}.
Encontrar una forma razonable de estimar estos parámetros será el objetivo de este capítulo (y más allá!), y eso es lo que veremos a continuación.

\section{Estimando los parámetros del modelo}

Nuestro objetivo principal es estimar los parámetros $\beta_0$ y $\beta_1$ del modelo lineal simple explicitado en \eqref{eq:modelo_lineal_simple_poblacional}. Para lograrlo es necesario suponer que contamos con una muestra de la población completa. Llamamos ${(x_i, y_i): i = 1, \ldots, n}$ a una muestra aleatoria de tamaño $n$ de la población, es decir, $x_i$ e $y_i$ son vectores con $n$ observaciones. Entonces reescribimos el modelo en términos de esta muestra:
\begin{equation}
y_i = \beta_0 + \beta_1 x_i + \mu_i,
\label{eq:modelo_lineal_simple}
\end{equation}
para todo $i$, donde $i = 1, \ldots, n$ es un índice que identifica a las observaciones (filas) en los datos. Podríamos tener $n$ personas, familias o empresas.

Esta ecuación, que sólo se diferencia de \eqref{eq:modelo_lineal_simple_poblacional} por ese pequeño subíndice $i$, es la versión \emph{muestral} del modelo lineal simple.
Recordemos que hemos supuesto que la variable dependiente $y_i$ es una función lineal de la variable independiente, $x_i$.
Entonces intentaremos buscar ---o \emph{estimar}--- una combinación de parámetros $\beta_0$ y $\beta_1$ que se ajuste a nuestros datos. Como es posible que los datos no tengan una relación lineal perfecta, el modelo incluye un término de error $\mu_i$. Este término representa todas las variables que afectan a $y_i$ que no son $x_i$.

Por ejemplo, podríamos tener un vector $y_i$ con los salarios de $n$ personas y modelarlo como una función lineal de los años de educación de esas personas, $x_i$. Resulta útil tener una base de datos concreta con estos vectores, por lo que usaremos una pequeña muestra de datos simulados.

\begin{knitrout}
\definecolor{shadecolor}{rgb}{0.969, 0.969, 0.969}\color{fgcolor}\begin{kframe}
\begin{alltt}
\hlcom{# Leer datos CSV de internet}
\hlstd{simdatos} \hlkwb{<-} \hlkwd{read.csv}\hlstd{(}
  \hlstr{"https://raw.githubusercontent.com/acarril/NER/master/datos/simdatos.csv"}\hlstd{)}
\hlcom{# Imprimir las primeras filas de los datos}
\hlkwd{head}\hlstd{(simdatos)}
\end{alltt}
\begin{verbatim}
##   y         x
## 1 1  4.199913
## 2 1  7.510634
## 3 1  2.125473
## 4 2  8.988857
## 5 2 10.243105
## 6 2 11.296823
\end{verbatim}
\end{kframe}
\end{knitrout}

\begin{Rbox}
\verb|read.csv()| permite leer bases de datos en formato CSV. Lo usamos para leer una base en línea y luego asignamos esos datos al objeto \verb|simdatos| (datos simulados). 
Usamos \verb|head()| para imprimir el encabezado de los datos, lo que permite hacernos una idea rápida de las variables y sus valores (puedes imprimir la base completa ejecutado \verb|simdatos|).
Esta base tiene solamente 30 observaciones.
\end{Rbox}

En esta base la variable \verb|y| representa ingreso por hora (en miles de pesos) y la variable \verb|x| representa años de educación.
Siempre es recomendable graficar los datos, ya que un gráfico entrega mucha información y revela patrones que son difíciles de ver en los datos.

\begin{knitrout}
\definecolor{shadecolor}{rgb}{0.969, 0.969, 0.969}\color{fgcolor}\begin{kframe}
\begin{alltt}
\hlkwd{ggplot}\hlstd{(simdatos,} \hlkwd{aes}\hlstd{(x, y))} \hlopt{+} \hlkwd{geom_point}\hlstd{()}
\end{alltt}
\end{kframe}

{\centering \includegraphics[width=4in,height=3in]{figs/unnamed-chunk-3-1} 

}



\end{knitrout}

\begin{Rbox}
\paq{ggplot2} es un paquete muy poderoso para crear gráficos de distintos tipos. Su principal función es \verb|ggplot()|, que permite definir los datos y las variables que usaremos (dentro de \verb|aes()|). Luego usamos \kw{funciones geom} para definir el tipo de gráfico; en este caso \verb|geom_point()| crea un gráfico de puntos.
\end{Rbox}

Vemos que existe una evidente relación lineal entre $x$ e $y$: a mayor nivel de educación parece haber mayor nivel de ingreso.
Esto es bueno para nosotros, ya que queremos ajustar los datos a un modelo lineal como el que escribimos en \eqref{eq:modelo_lineal_simple}.
Cabe preguntarse ahora cuál es la manera óptima de elegir los parámetros que determinan dicha relación lineal, es decir, ¿cómo podemos elegir $\beta_0$ (el intercepto) y $\beta_1$ (la pendiente) para que la predicción se ajuste lo mejor posible a los datos?

Podríamos elegir entre una infinita variedad de combinaciones de $\beta_0$ y $\beta_1$ para modelar los datos. Por ejemplo, el código de abajo simula 150 líneas con interceptos y pendientes ``razonables'':

\begin{knitrout}
\definecolor{shadecolor}{rgb}{0.969, 0.969, 0.969}\color{fgcolor}\begin{kframe}
\begin{alltt}
\hlkwd{set.seed}\hlstd{(}\hlnum{314}\hlstd{)}
\hlstd{modelos} \hlkwb{<-} \hlkwd{data.frame}\hlstd{(}
  \hlkwc{beta1} \hlstd{=} \hlkwd{runif}\hlstd{(}\hlnum{150}\hlstd{,} \hlopt{-}\hlnum{3}\hlstd{,} \hlnum{1}\hlstd{),}
  \hlkwc{beta2} \hlstd{=} \hlkwd{runif}\hlstd{(}\hlnum{150}\hlstd{,} \hlopt{-}\hlnum{1}\hlstd{,} \hlnum{1}\hlstd{)}
\hlstd{)}
\hlkwd{ggplot}\hlstd{(simdatos,} \hlkwd{aes}\hlstd{(x, y))} \hlopt{+}
  \hlkwd{geom_point}\hlstd{()} \hlopt{+}
  \hlkwd{geom_abline}\hlstd{(}
    \hlkwd{aes}\hlstd{(}\hlkwc{intercept} \hlstd{= beta1,} \hlkwc{slope} \hlstd{= beta2),}
    \hlkwc{data} \hlstd{= modelos,} \hlkwc{alpha} \hlstd{=} \hlnum{1}\hlopt{/}\hlnum{4}
  \hlstd{)}
\end{alltt}
\end{kframe}

{\centering \includegraphics[width=4in,height=3in]{figs/unnamed-chunk-4-1} 

}



\end{knitrout}

Si bien hay muchas líneas que claramente no se ajustan bien a los datos, varias otras sí lo hacen, y resulta difícil determinar cuál es ``la mejor'' (asumiendo que tal línea existe).
Es evidente que necesitamos un criterio riguroso para elegir los parámetros $\beta_0$ y $\beta_1$ de manera óptima.
Esto es lo que veremos a continuación.


\section{Mínimos Cuadrados Ordinarios}
\label{sec:MCO}

El método de Mínimos Cuadrados Ordinarios (MCO) es una respuesta directa a la pregunta que planteamos recién: ¿qué criterio conviene usar para elegir el modelo lineal que se ajuste ``mejor'' a los datos?
Geométricamente, el método de MCO equivale a minimizar la distancia vertical entre los puntos y la recta definida por el modelo.

\begin{knitrout}
\definecolor{shadecolor}{rgb}{0.969, 0.969, 0.969}\color{fgcolor}\begin{figure}

{\centering \includegraphics[width=4in,height=3in]{figs/MCO-1} 

}

\caption[Línea de ajuste (rojo) y residuos (azul) del modelo lineal simple]{Línea de ajuste (rojo) y residuos (azul) del modelo lineal simple}\label{fig:MCO}
\end{figure}


\end{knitrout}

Recordemos que estamos trabajando con la versión muestral del modelo lineal simple, es decir,
\begin{equation*}
y_i = \beta_0 + \beta_1 x_i + \mu_i.
\end{equation*}
Dada una muestra de datos (como \verb|simdatos|), llamamos $\hat\beta_0$ y $\hat\beta_1$ a los candidatos para los coeficientes de la estimación. Entonces la recta $y_i$ que estimamos está definida como
\begin{equation}
\hat y_i = \hat\beta_0 + \hat\beta_1 x_i.
\end{equation}

La diferencia entre $y_i$ (los datos) y $\hat y_i$ (nuestra predicción) corresponde a $\hat\mu_i$, el residuo de nuestra estimación para cada observación $i$: 
\begin{align}
\hat \mu_i &= y_i - \hat y_i \notag \\
           &= y_i - \hat\beta_0 + \hat\beta_1 x_i.
\label{eq:residuos}
\end{align}
Gráficamente, la distancia entre cada punto (los datos) y la recta (la predicción $\hat y_i$)  corresponde al residuo, $\hat\mu_i$, representada en el gráfico por las líneas rojas. 
Es decir que para cada observación $i$ tenemos una medida de la magnitud del error que esta elección de $\hat\beta_0$ y $\hat\beta_1$ produce.
El método de MCO corresponde a elegir $\hat\beta_0$ y $\hat\beta_1$ en \eqref{eq:residuos} tal que se minimice la suma de los residuos al cuadrado:\footnote{¿Por qué al cuadrado? Porque se asume que es irrelevante si el error es una subestimación o una sobreestimación, de forma que lo único que importa es la magnitud. ¿Por qué no se usa el valor absoluto entonces? Buena pregunta: \url{https://stats.stackexchange.com/q/46019/91358}. Básicamente, porque es algebraicamente más simple manejar valores al cuadrado, y no hace ninguna diferencia en el resultado final.}

\begin{align}
\min_{\hat\beta_0,\hat\beta_1} & \sum_{i=1}^n \hat\mu_i^2 \notag \\
\Leftrightarrow \min_{\hat\beta_0,\hat\beta_1} & \sum_{i=1}^n (y_i - \hat\beta_0 - \hat\beta_1 x_i)^2.
\label{eq:prob_min_MCO}
\end{align}

Al resolver este problema de minimización obtendremos los estimadores MCO:\footnote{La derivación paso a paso puede encontrarse, por ejemplo, en \textcite[cap. 2]{wooldridge_introductory_2016}.}

\begin{align}
\hat \beta_1 &= \sum_{i=1}^n \frac{\sum_{i=1}^n (x_i - \bar x)(y_i - \bar y)}{\sum_{i=1}^n (x_i - \bar x)^2} \notag \\
 &= \frac{\Cov (x,y)}{\Var(x)} \\
\hat \beta_0 &= \bar y - \hat\beta_1 \bar x
\end{align}
donde $\bar x$ e $\bar y$ son los promedios de $x_i$ e $y_i$, $\Var(x)$ es la varianza muestral de $x$ y $\Cov(x,y)$ es la covarianza muestral de $x$ e $y$.
Todo esto significa que para encontrar los estimadores MCO y obtener la línea de mejor ajuste simplemente tenemos que calcular cuatro cosas: $\bar y$, $\bar x$, $\Cov (x,y)$ y $\Var (x,y)$. ¡Hagámoslo ahora!

\begin{knitrout}
\definecolor{shadecolor}{rgb}{0.969, 0.969, 0.969}\color{fgcolor}\begin{kframe}
\begin{alltt}
\hlcom{# Indicar base de datos a usar}
\hlkwd{attach}\hlstd{(simdatos)}
\hlcom{# Promedios de 'x' e 'y'}
\hlstd{x.barra} \hlkwb{<-} \hlkwd{mean}\hlstd{(x)}
\hlstd{y.barra} \hlkwb{<-} \hlkwd{mean}\hlstd{(y)}
\hlcom{# Varianza de 'x' y covarianza entre 'x' e 'y'}
\hlstd{var.x} \hlkwb{<-} \hlkwd{var}\hlstd{(x)}
\hlstd{cov.xy} \hlkwb{<-} \hlkwd{cov}\hlstd{(x,y)}
\hlcom{# Calcular parámetros MCO}
\hlstd{beta1} \hlkwb{<-} \hlstd{cov.xy}\hlopt{/}\hlstd{var.x}
\hlstd{beta0} \hlkwb{<-} \hlstd{y.barra} \hlopt{-} \hlstd{beta1}\hlopt{*}\hlstd{x.barra}
\hlcom{# 'Desactivar' base de datos}
\hlkwd{detach}\hlstd{(simdatos)}
\hlcom{# Imprimir parámetros calculados}
\hlstd{beta0}
\end{alltt}
\begin{verbatim}
## [1] -1.185369
\end{verbatim}
\begin{alltt}
\hlstd{beta1}
\end{alltt}
\begin{verbatim}
## [1] 0.4311958
\end{verbatim}
\end{kframe}
\end{knitrout}
Vemos que es fácil calcular los estimadores MCO y en este caso $\hat \beta_0 = -1.19$ y $\hat \beta_1 = 0.43$.

Ahora podemos usar los parámetros estimados para graficar $\hat y_i$ sobre los datos:

\begin{knitrout}
\definecolor{shadecolor}{rgb}{0.969, 0.969, 0.969}\color{fgcolor}\begin{kframe}
\begin{alltt}
\hlcom{# Graficar línea de regresión MCO}
\hlkwd{ggplot}\hlstd{(simdatos,} \hlkwd{aes}\hlstd{(x,y))} \hlopt{+}
  \hlkwd{expand_limits}\hlstd{(}\hlkwc{x} \hlstd{=} \hlnum{0}\hlstd{,} \hlkwc{y} \hlstd{=} \hlopt{-}\hlnum{2.5}\hlstd{)} \hlopt{+}
  \hlkwd{geom_point}\hlstd{()} \hlopt{+}
  \hlkwd{geom_abline}\hlstd{(}
    \hlkwc{intercept} \hlstd{= beta0,} \hlkwc{slope} \hlstd{= beta1,}
    \hlkwc{color} \hlstd{=} \hlstr{"blue"}\hlstd{,} \hlkwc{show.legend} \hlstd{=} \hlnum{FALSE} \hlstd{)}
\end{alltt}
\end{kframe}

{\centering \includegraphics[width=4in,height=3in]{figs/unnamed-chunk-6-1} 

}



\end{knitrout}

Si bien el método que acabamos de usar para calcular $\hat\beta_0$ y $\hat\beta_1$ es útil para entender de manera más profunda qué es lo que estamos haciendo al estimar un modelo líneal por el método de MCO, no es demasiado breve. Podemos obtener los parámetros directamente usando la función \verb|lm()| de la siguiente forma:

\begin{knitrout}
\definecolor{shadecolor}{rgb}{0.969, 0.969, 0.969}\color{fgcolor}\begin{kframe}
\begin{alltt}
\hlkwd{lm}\hlstd{(y} \hlopt{~} \hlstd{x,} \hlkwc{data} \hlstd{= simdatos)}
\end{alltt}
\begin{verbatim}
## 
## Call:
## lm(formula = y ~ x, data = simdatos)
## 
## Coefficients:
## (Intercept)            x  
##     -1.1854       0.4312
\end{verbatim}
\end{kframe}
\end{knitrout}

La función \verb|lm()| viene de \eng{linear model}, y permite calcular rápidamente los parámetros de un modelo lineal (¡usamos una línea!). El primer argumento de la función es una fórmula de R, que no es exactamente lo mismo que una fórmula en el sentido usual de la palabra. Por el momento basta entender que a la izquierda de \verb|~| indicamos la variable dependiente, mientras que a la derecha indicamos las variables independientes. Entonces \verb|lm()| toma una fórmula como \verb|y~x| y la traduce automáticamente a algo como \verb|y = beta0 + beta1 * x|.

Una de las ventajas de usar \verb|lm()| es que podemos guardar sus resultados en un objeto, el que luego podemos manipular para extraer otro tipo de información útil del modelo. A continuación veremos cómo extraer, manipular e interpretar esta información, lo que nos dará una comprensión mucho más profunda de lo que estamos haciendo.


\section{Interpretando un modelo lineal}

Usaremos ahora la base \verb|wage2|, que es uno de los datos usados por \textcite{wooldridge_introductory_2016} y contiene información de ingresos (y otras variables) para 935 personas. \footnote{Esta base es un subconjunto de los datos del estudio de \textcite{blackburn_unobserved_1992}.}

\begin{knitrout}
\definecolor{shadecolor}{rgb}{0.969, 0.969, 0.969}\color{fgcolor}\begin{kframe}
\begin{alltt}
\hlkwd{library}\hlstd{(foreign)}
\hlcom{#ingresos <- read.dta("http://fmwww.bc.edu/ec-p/data/wooldridge/wage2.dta")}
\hlstd{ingresos} \hlkwb{<-} \hlkwd{read.dta}\hlstd{(}\hlstr{"datos/wage2.dta"}\hlstd{)}
\hlstd{ingresos} \hlkwb{<-} \hlkwd{as.data.frame}\hlstd{(ingresos[}\hlkwd{c}\hlstd{(}\hlstr{"wage"}\hlstd{,}\hlstr{"educ"}\hlstd{)])}
\end{alltt}
\end{kframe}
\end{knitrout}

El paquete \paq{foreign} nos permite leer bases de datos en formatos de otros programas. En este caso usamos la función \verb|foreign::read.dta()| para leer una base de datos de Stata.
Por el momento nos interesan dos variables: \verb|wage|, que es un vector de ingresos (en dólares menusales) y \verb|educ|, que es un vector con años de escolaridad de cada persona. La última línea guarda solamente esas dos variables, como marco de datos, en el objeto \verb|ingresos|.

Al cargar nuevos datos yo siempre recomiendo:
\begin{enumerate}
\item Imprimir el encabezado
\item Imprimir un resumen
\item Graficarlos
\end{enumerate}

Escribimos lo siguiente para lograr estas tres cosas:

\begin{knitrout}
\definecolor{shadecolor}{rgb}{0.969, 0.969, 0.969}\color{fgcolor}\begin{kframe}
\begin{alltt}
\hlkwd{head}\hlstd{(ingresos)}
\end{alltt}
\begin{verbatim}
##   wage educ
## 1  769   12
## 2  808   18
## 3  825   14
## 4  650   12
## 5  562   11
## 6 1400   16
\end{verbatim}
\begin{alltt}
\hlkwd{summary}\hlstd{(ingresos)}
\end{alltt}
\begin{verbatim}
##       wage             educ      
##  Min.   : 115.0   Min.   : 9.00  
##  1st Qu.: 669.0   1st Qu.:12.00  
##  Median : 905.0   Median :12.00  
##  Mean   : 957.9   Mean   :13.47  
##  3rd Qu.:1160.0   3rd Qu.:16.00  
##  Max.   :3078.0   Max.   :18.00
\end{verbatim}
\begin{alltt}
\hlkwd{ggplot}\hlstd{(ingresos,} \hlkwd{aes}\hlstd{(educ, wage))} \hlopt{+} \hlkwd{geom_point}\hlstd{()}
\end{alltt}
\end{kframe}

{\centering \includegraphics[width=4in,height=3in]{figs/unnamed-chunk-9-1} 

}



\end{knitrout}

Es evidente que estos datos (reales) no presentan una relación lineal tan evidente. Más adelante nos haremos cargo de este problema.

\subsubsection{Coeficientes}

Volvemos a usar \verb|lm()| como ya lo hemos hecho, esta vez definiendo una fórmula donde \verb|wage| es la variable dependiente y \verb|educ| es la variable independiente:
\begin{knitrout}
\definecolor{shadecolor}{rgb}{0.969, 0.969, 0.969}\color{fgcolor}\begin{kframe}
\begin{alltt}
\hlkwd{lm}\hlstd{(wage} \hlopt{~} \hlstd{educ,} \hlkwc{data} \hlstd{= ingresos)}
\end{alltt}
\begin{verbatim}
## 
## Call:
## lm(formula = wage ~ educ, data = ingresos)
## 
## Coefficients:
## (Intercept)         educ  
##      146.95        60.21
\end{verbatim}
\end{kframe}
\end{knitrout}


Los coeficientes estimados de un modelo lineal como este pueden interpretarse fácilmente, recordando que nuestra predicción es
\begin{equation*}
\hat y = \hat \beta_0 + \hat \beta_1 x.
\end{equation*}
Notar que esta ecuación es equivalente a una función lineal genérica, del tipo $y=a+bx$. Tener este simple hecho en cuenta es de gran ayuda para entender su interpretación.

El intercepto de un modelo lineal nos indica que si las variables independientes son todas iguales a 0, el modelo predice, en promedio, un valor de $\hat \beta_0$ para $\hat y$. En este ejemplo, si una persona tiene 0 años de educación el modelo predice un ingreso mensual promedio de 146.95 dólares.
Los coeficientes estimados para las variables independientes (solamente $\hat \beta_1$, en este caso) pueden interpretarse como el cambio promedio que un aumento de una unidad de $x$ tiene sobre $y$. Nuestra estimación indica que un año adicional de educación tiene un efecto promedio de aumentar en 60.21 dólares el ingreso de las personas. Si este coeficiente fuera negativo diríamos que un cambio en $x$ está correlacionado a una reducción de $y$.


\subsubsection{Errores estándar de los coeficientes}

Los parámetros estimados ---los coeficientes de MCO, en este caso--- son estadísticos muestrales que usamos para hacer inferencias de los parámetros poblacionales. Es importante tener en cuenta que son estos parámetros poblacionales los que nos interesan en realidad, pero como no podemos observarlos directamente, debemos inferirlos por medio de una muestra finita.

Es evidente que si tomáramos una muestra distinta a la actual, lo más probable es que estimaríamos parámetros diferentes. Si tomaras datos de otras 935 personas (o cualquier número, en realidad) es muy posible que al aplicar el mismo modelo obtuviéramos valores de $\beta_0$ y $\beta_1$ distintos de 146.95 y 60.21. Además es muy probable que nuevamente ninguno de los dos sea igual al verdadero parámetro poblacional, que es el que nos interesa. Si continuáramos este proceso de tomar diferentes muestras y estimar los mismos parámetros una y otra vez, veríamos que la frecuencia relativa de las estimaciones obtenidas sigue una distribución de probabilidad, y por el Teorema Central del Límite sabemos que es probable que esta distribución sea normal. Entonces es importante para nuestra estimación que cuantifiquemos la cantidad de incertidumbre en esta distribución poblacional desconocida. Ahí es donde entra el concepto de error estándar, que corresponde a nuestra estimación de la desviación estándar de la distribución de estos muestreos. En términos intuitivos, es una medida de la incertidumbre de $\hat\beta$.

Encerramos \verb|lm()| dentro de \verb|summary()| para obtener información más detallada de las estimaciones del modelo, incluyendo estimaciones para los errores estándar de los coeficientes:

\begin{knitrout}
\definecolor{shadecolor}{rgb}{0.969, 0.969, 0.969}\color{fgcolor}\begin{kframe}
\begin{alltt}
\hlkwd{summary}\hlstd{(}\hlkwd{lm}\hlstd{(wage} \hlopt{~} \hlstd{educ,} \hlkwc{data} \hlstd{= ingresos))}
\end{alltt}
\begin{verbatim}
## 
## Call:
## lm(formula = wage ~ educ, data = ingresos)
## 
## Residuals:
##    Min     1Q Median     3Q    Max 
## -877.4 -268.6  -38.4  207.0 2148.3 
## 
## Coefficients:
##             Estimate Std. Error t value Pr(>|t|)    
## (Intercept)   146.95      77.71    1.89    0.059 .  
## educ           60.21       5.69   10.57   <2e-16 ***
## ---
## Signif. codes:  0 '***' 0.001 '**' 0.01 '*' 0.05 '.' 0.1 ' ' 1
## 
## Residual standard error: 382 on 933 degrees of freedom
## Multiple R-squared:  0.107,	Adjusted R-squared:  0.106 
## F-statistic:  112 on 1 and 933 DF,  p-value: <2e-16
\end{verbatim}
\end{kframe}
\end{knitrout}

En el ejemplo, ...

%todo: explicar mejor todos los summary de ols

\begin{knitrout}
\definecolor{shadecolor}{rgb}{0.969, 0.969, 0.969}\color{fgcolor}\begin{kframe}
\begin{alltt}
\hlkwd{ggplot}\hlstd{(ingresos,} \hlkwd{aes}\hlstd{(educ, wage))} \hlopt{+}
  \hlkwd{geom_point}\hlstd{()} \hlopt{+}
  \hlkwd{geom_smooth}\hlstd{(}\hlkwc{method} \hlstd{=} \hlstr{"lm"}\hlstd{)}
\end{alltt}
\end{kframe}

{\centering \includegraphics[width=4in,height=3in]{figs/unnamed-chunk-13-1} 

}



\end{knitrout}


\subsection{Extraer estadísticos de una regresión}

Para guardar el modelo asignamos \verb|lm()| a un objeto:

\begin{knitrout}
\definecolor{shadecolor}{rgb}{0.969, 0.969, 0.969}\color{fgcolor}\begin{kframe}
\begin{alltt}
\hlstd{modelo} \hlkwb{<-} \hlkwd{lm}\hlstd{(wage} \hlopt{~} \hlstd{educ,} \hlkwc{data} \hlstd{= ingresos)}
\end{alltt}
\end{kframe}
\end{knitrout}

El objeto \verb|modelo| no solo incluye los coeficientes estimados, si no que una serie de otros elementos que son muy útiles ---veremos ahora por qué. En estricto rigor \verb|modelo| es lo que R llama una lista, y contiene una serie de otros objetos. Podemos imprimir los nombres de estos objetos escribiendo
\begin{knitrout}
\definecolor{shadecolor}{rgb}{0.969, 0.969, 0.969}\color{fgcolor}\begin{kframe}
\begin{alltt}
\hlkwd{names}\hlstd{(modelo)}
\end{alltt}
\begin{verbatim}
##  [1] "coefficients"  "residuals"     "effects"       "rank"         
##  [5] "fitted.values" "assign"        "qr"            "df.residual"  
##  [9] "xlevels"       "call"          "terms"         "model"
\end{verbatim}
\end{kframe}
\end{knitrout}

Para acceder a los objetos guardados usamos la notación de \verb|$|, como es usual en R. Por ejemplo, para imprimir los coeficientes del modelo escribimos
\begin{knitrout}
\definecolor{shadecolor}{rgb}{0.969, 0.969, 0.969}\color{fgcolor}\begin{kframe}
\begin{alltt}
\hlstd{modelo}\hlopt{$}\hlstd{coefficients}
\end{alltt}
\begin{verbatim}
## (Intercept)        educ 
##         147          60
\end{verbatim}
\end{kframe}
\end{knitrout}
También existen funciones para acceder directamente a los objetos. Por ejemplo,
\begin{knitrout}
\definecolor{shadecolor}{rgb}{0.969, 0.969, 0.969}\color{fgcolor}\begin{kframe}
\begin{alltt}
\hlkwd{coefficients}\hlstd{(modelo)}
\end{alltt}
\begin{verbatim}
## (Intercept)        educ 
##         147          60
\end{verbatim}
\end{kframe}
\end{knitrout}
En cualquier caso, el resultado es un vector cuyos elementos tienen nombres: el intercepto ($\beta_0$) es \texttt{(Intercept)} y el nombre del primer coeficiente ($\beta_1$) es el nombre de la variable $x$, es decir, \texttt{educ}. Usando esto podemos acceder a elementos específicos del vector usando sus nombres o sus índices, como se mencionó en [ref]:
\begin{knitrout}
\definecolor{shadecolor}{rgb}{0.969, 0.969, 0.969}\color{fgcolor}\begin{kframe}
\begin{alltt}
\hlkwd{coef}\hlstd{(modelo)[}\hlnum{1}\hlstd{]}
\end{alltt}
\begin{verbatim}
## (Intercept) 
##         147
\end{verbatim}
\begin{alltt}
\hlkwd{coef}\hlstd{(modelo)[}\hlstr{"educ"}\hlstd{]}
\end{alltt}
\begin{verbatim}
## educ 
##   60
\end{verbatim}
\end{kframe}
\end{knitrout}

\subsection{Predicciones y residuos}

La predicción de un modelo, a veces llamada valores ajustados, es simplemente el vector $\hat y_i$. Teniendo los valores ajustados podemos calcular también los residuos del modelo, $\mu_i$:
\begin{align}
\hat y_i &= \hat \beta_0 + \hat\beta_1 x_i \label{eq:prediccion} \\
\hat \mu_i &= y_i - \hat y_i \label{eq:residuos}
\end{align}
Podemos calcular fácilmente ambos vectores con los elementos que tenemos:
\begin{knitrout}
\definecolor{shadecolor}{rgb}{0.969, 0.969, 0.969}\color{fgcolor}\begin{kframe}
\begin{alltt}
\hlstd{ingresos}\hlopt{$}\hlstd{prediccion} \hlkwb{<-} \hlkwd{coef}\hlstd{(modelo)[}\hlnum{1}\hlstd{]} \hlopt{+} \hlkwd{coef}\hlstd{(modelo)[}\hlnum{2}\hlstd{]} \hlopt{*} \hlstd{ingresos}\hlopt{$}\hlstd{educ}
\hlstd{ingresos}\hlopt{$}\hlstd{residuos} \hlkwb{<-} \hlstd{ingresos}\hlopt{$}\hlstd{wage} \hlopt{-} \hlstd{ingresos}\hlopt{$}\hlstd{prediccion}
\end{alltt}
\end{kframe}
\end{knitrout}

Alternativamente, podemos calcular los valores predichos y los residuos del modelo usando funciones específicas para ello:
\begin{knitrout}
\definecolor{shadecolor}{rgb}{0.969, 0.969, 0.969}\color{fgcolor}\begin{kframe}
\begin{alltt}
\hlstd{modelo.sim} \hlkwb{<-} \hlkwd{lm}\hlstd{(y} \hlopt{~} \hlstd{x,} \hlkwc{data} \hlstd{= simdatos)}
\hlstd{simdatos}\hlopt{$}\hlstd{prediccion} \hlkwb{<-} \hlkwd{fitted}\hlstd{(modelo.sim)}
\hlstd{simdatos}\hlopt{$}\hlstd{residuos} \hlkwb{<-} \hlkwd{resid}\hlstd{(modelo.sim)}
\end{alltt}
\end{kframe}
\end{knitrout}

Podemos graficar nuestra predicción sobre los datos para inspeccionar visualmente cómo se ve. Para esto volvemos a usar \paq{ggplot2}:
\begin{knitrout}
\definecolor{shadecolor}{rgb}{0.969, 0.969, 0.969}\color{fgcolor}\begin{kframe}
\begin{alltt}
\hlkwd{ggplot}\hlstd{(ingresos,} \hlkwd{aes}\hlstd{(educ, wage))} \hlopt{+}
  \hlkwd{geom_point}\hlstd{()} \hlopt{+}
  \hlkwd{geom_line}\hlstd{(}\hlkwd{aes}\hlstd{(}\hlkwc{y} \hlstd{= prediccion),} \hlkwc{color} \hlstd{=} \hlstr{"blue"}\hlstd{)}
\end{alltt}
\end{kframe}

{\centering \includegraphics[width=4in,height=3in]{figs/unnamed-chunk-21-1} 

}



\end{knitrout}

También resulta interesante analizar los residuos, que son ``la otra cara de la moneda''. Si bien los valores predichos nos muestran el patrón que el modelo ha sido capaz de capturar, los residuos nos mostrarán lo que el modelo ha omitido. Recordar que los residuos son simplemente la diferencia entre el valor predicho y el real, como se indica en \eqref{eq:residuos}. Graficamos los residuos escribiendo

\begin{knitrout}
\definecolor{shadecolor}{rgb}{0.969, 0.969, 0.969}\color{fgcolor}\begin{kframe}
\begin{alltt}
\hlkwd{ggplot}\hlstd{(ingresos,} \hlkwd{aes}\hlstd{(educ, residuos))} \hlopt{+}
  \hlkwd{geom_hline}\hlstd{(}\hlkwc{yintercept} \hlstd{=} \hlnum{0}\hlstd{,} \hlkwc{color} \hlstd{=} \hlstr{"gray"}\hlstd{)} \hlopt{+}
  \hlkwd{geom_point}\hlstd{()}
\end{alltt}
\end{kframe}

{\centering \includegraphics[width=4in,height=3in]{figs/unnamed-chunk-22-1} 

}



\end{knitrout}

\begin{knitrout}
\definecolor{shadecolor}{rgb}{0.969, 0.969, 0.969}\color{fgcolor}\begin{kframe}
\begin{alltt}
\hlkwd{ggplot}\hlstd{(simdatos,} \hlkwd{aes}\hlstd{(x, residuos))} \hlopt{+}
  \hlkwd{geom_hline}\hlstd{(}\hlkwc{yintercept} \hlstd{=} \hlnum{0}\hlstd{,} \hlkwc{color} \hlstd{=} \hlstr{"gray"}\hlstd{)} \hlopt{+}
  \hlkwd{geom_point}\hlstd{()}
\end{alltt}
\end{kframe}

{\centering \includegraphics[width=4in,height=3in]{figs/unnamed-chunk-23-1} 

}



\end{knitrout}

Además de estos elementos directamente accesibles guardados en \verb|modelo|, existen una serie de funciones que podemos aplicar al objeto para realizar otros cálculos o resumir información útil. Por ejemplo, es muy común usar \verb|summary()| para obtener un resumen de información importante del modelo estimado:
\begin{knitrout}
\definecolor{shadecolor}{rgb}{0.969, 0.969, 0.969}\color{fgcolor}\begin{kframe}
\begin{alltt}
\hlkwd{summary}\hlstd{(modelo)}
\end{alltt}
\begin{verbatim}
## 
## Call:
## lm(formula = wage ~ educ, data = ingresos)
## 
## Residuals:
##    Min     1Q Median     3Q    Max 
## -877.4 -268.6  -38.4  207.0 2148.3 
## 
## Coefficients:
##             Estimate Std. Error t value Pr(>|t|)    
## (Intercept)   146.95      77.71    1.89    0.059 .  
## educ           60.21       5.69   10.57   <2e-16 ***
## ---
## Signif. codes:  0 '***' 0.001 '**' 0.01 '*' 0.05 '.' 0.1 ' ' 1
## 
## Residual standard error: 382 on 933 degrees of freedom
## Multiple R-squared:  0.107,	Adjusted R-squared:  0.106 
## F-statistic:  112 on 1 and 933 DF,  p-value: <2e-16
\end{verbatim}
\end{kframe}
\end{knitrout}

\section{Valor esperado de los estimadores}

Recordemos el hecho que los estimadores MCO ($\hat\beta_0$ y $\hat\beta_1$, o en general $\hat\beta$) tienen una distribución de probabilidad.
Esto ocurre porque si tomáramos distintas muestras de tamaño $n$, obtendríamos distintos valores para $\hat\beta$.
Entonces nuestra estimación de $\hat\beta$ es solo una realización de infinidad de posibilidades, como se muestra en la \autoref{fig:parametros_distribucion}.

\begin{figure}[htb]
  \centering
  \begin{tikzpicture}
    \begin{axis}[
    no markers, domain=0:8, samples=100,
    axis lines*=left, xlabel=$x$, ylabel=$y$,
    every axis y label/.style={at=(current axis.above origin),anchor=south},
    every axis x label/.style={at=(current axis.right of origin),anchor=west},
    height=4cm, width=6cm,
    xtick={3},
    xticklabels={$\hat\beta$},
    ytick=\empty,
    enlargelimits=false, clip=false, axis on top,
    grid = major
    ]
    \addplot [very thick, black] {gauss(4,1)};
    \end{axis}
  \end{tikzpicture}
  \caption{Los parámetros estimados provienen de una distribución muestral}
  \label{fig:parametros_distribucion}
\end{figure}%

Parece natural querer que nuestros estimadores estén centrados en torno al verdadero parámetro $\beta$, es decir,
\begin{equation}
\E(\hat \beta) = \beta.
\end{equation}
Un estimador que cumple con esta propiedad se llama insesgado.
La \autoref{fig:estimadores_sesgo} muestra la distribución de un estimador insesgado en azul, mientras que la distribución de un estimador sesgado está graficada en rojo.
Notar que aunque ambos tienen igual varianza, el primero está centrado en torno al verdadero valor de $\beta$, y por lo tanto es insesgado.


\begin{figure}[htb]
\centering
\begin{subfigure}{.5\textwidth}
  \centering
  \begin{tikzpicture}
    \begin{axis}[
    no markers, domain=0:10, samples=100,
    axis lines*=left, xlabel=$x$, ylabel=$y$,
    every axis y label/.style={at=(current axis.above origin),anchor=south},
    every axis x label/.style={at=(current axis.right of origin),anchor=west},
    height=4cm, width=6cm,
    xtick={4},
    xticklabels={$\beta$},
    ytick=\empty,
    enlargelimits=false, clip=false, axis on top,
    grid = major
    ]
    \addplot [very thick,cyan!50!black] {gauss(4,1)};
    \addplot [very thick,magenta!50!black] {gauss(6,1)};
    \end{axis}
  \end{tikzpicture}
  \caption{Estimador insesgado (azul) vs. estimador sesgado (rojo)}
  \label{fig:estimadores_sesgo}
\end{subfigure}%
\begin{subfigure}{.5\textwidth}
  \centering
  \begin{tikzpicture}
    \begin{axis}[
    no markers, domain=0:10, samples=100,
    axis lines*=left, xlabel=$x$, ylabel=$y$,
    every axis y label/.style={at=(current axis.above origin),anchor=south},
    every axis x label/.style={at=(current axis.right of origin),anchor=west},
    height=4cm, width=6cm,
    xtick={5},
    xticklabels={$\beta$},
    ytick=\empty,
    enlargelimits=false, clip=false, axis on top,
    grid = major
    ]
    \addplot [very thick,cyan!50!black] {gauss(5,0.5)};
    \addplot [very thick,magenta!50!black] {gauss(5,1.2)};
    \end{axis}
  \end{tikzpicture}
  \caption{Estimadores insesgados con distinta varianza}
  \label{fig:estimadores_varianza}
\end{subfigure}
\caption{Heterocedasticidad}
\label{fig:estimadores_OLS_propiedades}
\end{figure}

Además de requerir que el estimador se centre en torno al verdadero valor del parámetro, también queremos saber qué tan lejos podemos esperar que $\hat\beta$ se aleje de $\beta$ en promedio. Esto nos permitirá elegir el estimador más eficiente, es decir, de entre todos los estimadores insesgados elegir aquél que tenga la menor varianza.
La \autoref{fig:estimadores_varianza} muestra la distribución de dos estimadores insesgados. Sin embargo, el estimador representado por la curva roja tiene relativamente mayor varianza que el representado por la azul.

Nuestro objetivo es obtener ambas propiedades para nuestros estimadores MCO.
Sin embargo, estas propiedades sólo existirán en la medida que se cumplan algunos supuestos. En esta sección veremos cuáles son estos supuestos y qué propiedades entregan.

Es importante tener en cuenta que las propiedades estadísticas no tienen nada que ver con una muestra en particular, si no que es una propiedad que los estimadores cumplirán cuando se realice un muestreo aleatorio repetidamente.


\paragraph{RLS.1: Lineal en parámetros} Esto corresponde al supuesto fundamental del modelo lineal, donde asumimos que la población sigue una función lineal:
\begin{equation}
y = \beta_0 + \beta_1 x + \mu.
\end{equation}
Nuestro objetivo es estimar $\beta_0$ y (en especial) $\beta_1$ suponiendo que los datos se relacionan de forma lineal en parámetros y que $x$, $y$ y $\mu$ son variables aleatorias.

\paragraph{RLS.2: Muestreo aleatorio} Suponemos que tenemos una muestra aleatoria de tamaño $n$ de la población total, de forma que en términos muestrales el modelo lineal es
\begin{equation}
y_i = \beta_0 + \beta_1 x_i + \mu_i, \quad i = 1,\ldots,n.
\end{equation}
Sí, es lo mismo que arriba, pero con el subíndice.

\paragraph{RLS.3: Variación en las variables independientes}
Necesitamos que la (o las) variables independientes del modelo tengan algo de variación.
Este supuesto es fácil, pero necesario matemáticamente.
Por ejemplo, para estimar el modelo de Mincer es necesario tener una muestra donde la gente tenga distintos niveles de educación. Si todos tienen exactamente los mismos años de escolaridad entonces es imposible determinar la relación entre años de escolaridad e ingresos.

\paragraph{RLS.4: Media condicional cero}
El error $u$ debe tener un valor esperado igual a cero, condicional a los valores de las variables independientes. En una muestra aleatoria esto significa que

\begin{equation}
E(\mu_i | x_i) = 0 \quad \forall\; i = 1,\ldots, n
\end{equation}

Usando los supuestos RLS.1 a RLS.4 se puede demostrar que los estimadores MCO son insesgados, es decir,
\begin{equation}
\E(\hat\beta_0) = \beta_0 \text{ y } \E(\hat\beta_1) = \beta_1.
\end{equation}

En general, si cualquiera de los supuestos no se cumple entonces perderemos la propiedad de insesgamiento. RLS.1 requiere que $y$ y $x$ estén relacionados linealmente, lo que ciertamente puede no cumplirse. Hay que recordar, sin embargo, que la relación es solamente lineal en \emph{parámetros}, no en variables, por lo que ciertamente podemos transformar $x$ e $y$ para capturar relaciones no lineales más interesantes.
RLS.2 asume que nuestros datos corresponden a una muestra aleatoria de la población, lo que claramente no siempre se cumple en datos de corte transversal. Por ejemplo, es posible que solamente tengamos datos de micro y pequeñas empresas, o que solamente observemos jefas de hogar de los quintiles más pobres.

RLS.3 es extremadamente básico y con mucha seguridad se cumplirá en cualquier tipo de datos. Sin embargo, RSL.4 ciertamente representa un problema potencial. Si no se cumple que la media condicional del error es 0, esto es equivalente a decir que existe alguna correlación entre $x$ y $\mu$.
Una razón por la que esto puede ocurrir es porque dejamos de incluir una variable relevante en nuestro modelo, de forma que esta afecta a $x$ por medio de $\mu$.

\section{Varianza de los estimadores}

Además de asegurarnos que la distribución muestral de $\hat\beta$ esté centrada alrededor de $\beta$ ---es decir, que $\hat\beta$ es insesgado---, es importante saber qué tan dispersa es esta distribución de $\hat\beta$.
La \autoref{fig:estimadores_varianza} muestra el caso de dos estimadores insesgados donde uno tiene una distribución con mayor varianza que el otro.
Parece lógico querer elegir, de entre todos los estimadores insesgados, aquel con menor varianza (ie. el azul).

Para poder calcular la varianza de los estimadores MCO es necesario agregar un último supuesto:

\paragraph{RLS.5: Homocedasticidad}\index{homocedasticidad} 
Asumimos que el error $\mu$ tiene la misma varianza condicional para todos los valores de la variable independiente. Es decir,
\begin{equation}
\Var(\mu | x) = \sigma^2.
\end{equation}

Usando los supuestos SLR.1 a SLR.5 podemos demostrar que

\begin{align}
\Var(\hat\beta_0) &= \frac{\sigma^2/n \sum_{i=1}^n x_i^2}{\sum_{i=1}^n (x_i - \overline x)^2} \label{eq:var_hat_beta0} \\
\Var(\hat\beta_1) &= \frac{\sigma^2}{\sum_{i=1}^n (x_i - \overline x)^2}.\label{eq:var_hat_beta1}
\end{align}

Usualmente nos interesará $\Var(\hat\beta_1)$.
Vemos que esta varianza depende positivamente de la varianza del error $\sigma^2$. Esto tiene sentido, ya que a mayor varianza de los factores no observables que afectan $y$, es más difícil estimar con precisión $\beta_1$.
Por otro lado, $\Var(\hat\beta_1)$ depende negativamente de la variación total de $x$, representada por $SST_x = \sum_{i=1}^n (x_i - \overline x)^2$. Esto es así porque entre más dispersos estén los valores de la variable independiente, más fácil será detectar la relación entre $\E(y|x)$ y $x$.
Además, a medida que el tamaño muestral aumente, también lo hará la variación total en $x_i$, de forma que una muestra más grande siempre resultará en una menor varianza de $\hat\beta_1$.

Cuando $\Var(\mu | x)$ depende de $x$ (cambia con $x$) decimos que el término de error presenta \kw{heterocedasticidad}, o varianza no constante.
La heterocedasticidad es un problema que es más difícil de pronunciar que de entender: ocurre cuando la dispersión de una variable dependiente no es constante para distintos valores de la variable independiente.

La \autoref{fig:heterocedasticidad} muestra datos heterocedásticos para el caso de una regresión bivariada, tanto para una variable dependiente discreta como continua.
En el eje $f(y|x)$ se grafica la densidad del término de error, la que en este ejemplo claramente claramente disminuye a medida que $x$ es mayor.

\begin{figure}[htb]
\pgfmathsetseed{112}
\centering
\begin{subfigure}{.5\textwidth}
\makeatletter
        \pgfdeclareplotmark{dot}
        {%
            \fill circle [x radius=0.08, y radius=0.32];
        }%
\makeatother
  \centering
  \begin{tikzpicture}[ % Define Normal Probability Function
declare function={
            normal(\x,\m,\s) = 1/(2*\s*sqrt(pi))*exp(-(\x-\m)^2/(2*\s^2));
        },
    declare function={invgauss(\a,\b) = sqrt(-2*ln(\a))*cos(deg(2*pi*\b));}
       ]
\begin{axis}[
    %no markers,
    domain=0:12,
    zmin=0, zmax=1,
    xmin=0, xmax=3,
    samples=200,
   samples y=0,
    view={40}{30},
    axis lines=middle,
    enlarge y limits=false,
    xtick={0.5,1.5,2.5},
    xmajorgrids,
    xticklabels={},
    ytick=\empty,
%   xticklabels={$x_1$, $x_2$, $x_3$},
    ztick=\empty,
    xlabel=$x$, xlabel style={at={(rel axis cs:1,0,0)}, anchor=west},
    ylabel=$y$, ylabel style={at={(rel axis cs:0,1,0)}, anchor=south west},
    zlabel=$f(y|x)$, zlabel style={at={(rel axis cs:0,0,1)}, anchor=south},
    set layers, mark=cube
  ]

\pgfplotsinvokeforeach{0.5,1.5,2.5}{
\addplot3 [draw=none, fill=black, opacity=0.25, only marks, mark=dot, mark layer=like plot, samples=30, domain=0.1:2.9, on layer=axis background] (#1, {1.5*(#1-0.5)+3+invgauss(rnd,rnd)*#1}, 0);
}
\addplot3 [samples=2, samples y=0, domain=0:3] (x, {1.5*(x-0.5)+3}, 0);
\addplot3 [blue!50, thick] (0.5, x, {normal(x, 3, 0.5)});
\addplot3 [blue!50, thick] (1.5, x, {normal(x, 4.5, 1)});
\addplot3 [blue!50, thick] (2.5, x, {normal(x, 6, 1.5)});

\pgfplotsextra{
\begin{pgfonlayer}{axis background}
\draw [gray, on layer=axis background] (0.5, 3, 0) -- (0.5, 3, {normal(0,0,0.5)}) (0.5,0,0) -- (0.5,12,0)
    (1.5, 4.5, 0) -- (1.5, 4.5, {normal(0,0,1)}) (1.5,0,0) -- (1.5,12,0)
    (2.5, 6, 0) -- (2.5, 6, {normal(0,0,1.5)}) (2.5,0,0) -- (2.5,12,0);

\end{pgfonlayer}
}
\end{axis}

\end{tikzpicture}
  \caption{Variable dependiente discreta}
  \label{fig:heterocedasticidad_x_discreta}
\end{subfigure}%
\makeatletter
        \pgfdeclareplotmark{dot}
        {%
            \fill circle [x radius=0.02, y radius=0.08];
        }%
\makeatother
\begin{subfigure}{.5\textwidth}
  \centering
  \begin{tikzpicture}[ % Define Normal Probability Function
declare function={
            normal(\x,\m,\s) = 1/(2*\s*sqrt(pi))*exp(-(\x-\m)^2/(2*\s^2));
        },
    declare function={invgauss(\a,\b) = sqrt(-2*ln(\a))*cos(deg(2*pi*\b));}
       ]
\begin{axis}[
    %no markers,
    domain=0:12,
    zmin=0, zmax=1,
    xmin=0, xmax=3,
    samples=200,
   samples y=0,
    view={40}{30},
    axis lines=middle,
    enlarge y limits=false,
    xtick={0.5,1.5,2.5},
    xmajorgrids,
    xticklabels={},
    ytick=\empty,
%   xticklabels={$x_1$, $x_2$, $x_3$},
    ztick=\empty,
    xlabel=$x$, xlabel style={at={(rel axis cs:1,0,0)}, anchor=west},
    ylabel=$y$, ylabel style={at={(rel axis cs:0,1,0)}, anchor=south west},
    zlabel=$f(y|x)$, zlabel style={at={(rel axis cs:0,0,1)}, anchor=south},
    set layers, mark=cube
  ]

\addplot3 [gray!50, only marks, mark=dot, mark layer=like plot, samples=200, domain=0.1:2.9, on layer=axis background] (x, {1.5*(x-0.5)+3+invgauss(rnd,rnd)*x}, 0);
\addplot3 [samples=2, samples y=0, domain=0:3] (x, {1.5*(x-0.5)+3}, 0);
\addplot3 [blue!50, thick] (0.5, x, {normal(x, 3, 0.5)});
\addplot3 [blue!50, thick] (1.5, x, {normal(x, 4.5, 1)});
\addplot3 [blue!50, thick] (2.5, x, {normal(x, 6, 1.5)});

\pgfplotsextra{
\begin{pgfonlayer}{axis background}
\draw [gray, on layer=axis background] (0.5, 3, 0) -- (0.5, 3, {normal(0,0,0.5)}) (0.5,0,0) -- (0.5,12,0)
    (1.5, 4.5, 0) -- (1.5, 4.5, {normal(0,0,1)}) (1.5,0,0) -- (1.5,12,0)
    (2.5, 6, 0) -- (2.5, 6, {normal(0,0,1.5)}) (2.5,0,0) -- (2.5,12,0);

\end{pgfonlayer}
}
\end{axis}


\end{tikzpicture}
  \caption{Variable dependiente continua}
  \label{fig:heterocedasticidad_x_continua}
\end{subfigure}
\caption{Heterocedasticidad}
\label{fig:heterocedasticidad}
\end{figure}

Podemos generar fácilmente un conjunto de datos que presenten heterocedasticidad:

\begin{knitrout}
\definecolor{shadecolor}{rgb}{0.969, 0.969, 0.969}\color{fgcolor}\begin{kframe}
\begin{alltt}
\hlkwd{set.seed}\hlstd{(}\hlnum{314}\hlstd{)}
\hlstd{n} \hlkwb{<-} \hlnum{256}                           \hlcom{# Número de observaciones}
\hlstd{x} \hlkwb{<-} \hlstd{(}\hlnum{1}\hlopt{:}\hlstd{n)}\hlopt{/}\hlstd{n}                       \hlcom{# Valores de `x`}
\hlstd{e} \hlkwb{<-} \hlkwd{rnorm}\hlstd{(n,} \hlkwc{sd}\hlstd{=}\hlnum{1}\hlstd{)}                \hlcom{# Valores aleatorios de una normal}
\hlstd{i} \hlkwb{<-} \hlkwd{order}\hlstd{(}\hlkwd{runif}\hlstd{(n,} \hlkwc{max}\hlstd{=}\hlkwd{dnorm}\hlstd{(e)))} \hlcom{# Ubicamos los más grandes al final}
\hlstd{y} \hlkwb{<-} \hlnum{1} \hlopt{+} \hlnum{5} \hlopt{*} \hlstd{x} \hlopt{+} \hlstd{e[}\hlkwd{rev}\hlstd{(i)]}         \hlcom{# Generamos `y` con el error `e`.}
\hlstd{modelo} \hlkwb{<-} \hlkwd{lm}\hlstd{(y} \hlopt{~} \hlstd{x)}                \hlcom{# Guardamos el modelo lineal}
\hlkwd{plot}\hlstd{(x, y)}
\hlkwd{abline}\hlstd{(}\hlkwd{coef}\hlstd{(modelo),} \hlkwc{col} \hlstd{=} \hlstr{"Red"}\hlstd{)}
\end{alltt}
\end{kframe}

{\centering \includegraphics[width=4in,height=3in]{figs/unnamed-chunk-25-1} 

}



\end{knitrout}

En presencia de heterocedasticidad una estimación por MCO seguirá entregando coeficientes consistentes e insesgados. Sin embargo, no podremos estimar correctamente la matriz de varianzas-covarianzas, lo que producirá que los errores estándar de los coeficientes estén sesgados. Esto conduce a errores en tests de inferencia, como (por ejemplo) al determinar si un coeficiente es significativo.

\subsection{Estimando la varianza del error}

Usando los supuesos de regresión lineal simple obtenemos la varianza de $\hat\beta_0$ y $\hat\beta_1$ indicadas en \eqref{eq:var_hat_beta0} y \eqref{eq:var_hat_beta1}.
El problema es que estas ecuaciones contienen requieren que conozcamos la varianza del error $\sigma^2$, la que casi siempre es desconocida.
Entonces nuestro objetivo ahora es usar nuestros datos para estimar $\sigma^2$, y así poder estimar la varianza de los parámetros.

Usando los supuestos RLS.1 a RLS.5 es posible demostrar que
\begin{equation}
\E(\hat\sigma^2) = \sigma^2.
\end{equation}
Entonces podríamos definir definir el estimador $1/n \sum_{i=1}^n \hat\mu_i^2$ para . Lamentablemente este estimador es sesgado, ya que no toma en cuenta dos restricciones que deben satisfacer los residuos de MCO:
\begin{align*}
\sum_{i=1}^n \hat\mu_i &= 0 \\
\sum_{i=1}^n x_i\hat\mu_i &= 0
\end{align*}
Estas restricciones vienen de las condiciones de primer orden del problema de minimización de MCO planteado en \eqref{eq:prob_min_MCO}.
Una manera de entender cómo influyen estas restricciones al estimador planteado recién es pensando que si conociéramos $n-2$ residuos, siempre podríamos obtener los últimos 2 usando estas restricciones (piensa por qué).
Esto implica que si bien tenemos $n$ grados de libertad para los errores, existen $n-2$ grados de libertad para los residuos de MCO.
Entonces el estimador insesgado de $\sigma^2$ corrige el estimador propuesto recién por este cambio en grados de libertad:
\begin{equation}
\hat\sigma^2 = \frac{1}{n-2} \sum_{i=1}^n \hat \mu_i^2.
\end{equation}
Este estimador a veces es denotado por $S^2$.
Una vez que obtenemos $\hat\sigma^2$ podemos usarlo en las ecuaciones \eqref{eq:var_hat_beta0} y \eqref{eq:var_hat_beta1} para calcular $\Var(\hat\beta_0)$ y $\Var(\hat\beta_1)$.

\subsection{Error estándar (de los estimadores)}

Parece natural pensar que un estimador para la desviación estándar del error es
\begin{equation}
\hat\sigma = \sqrt{\hat\sigma^2}.
\end{equation}
A este término se le conoce como el \kw{error estándar de la regresión}. Lamentablemente $\hat\sigma$ no es un estimador insesgado de $\sigma$, pero es consistente: a medida que la cantidad de datos aumenta, el estimador converge en probabilidad al verdadero parámetro $\sigma$.
Esta situación está representada en la \autoref{fig:estimador_sesgado_consistente}.

\begin{figure}[htb]
  \centering
  \begin{tikzpicture}
    \begin{axis}[
    no markers, domain=0:8, samples=100,
    axis lines*=left, xlabel=$x$, ylabel=$y$,
    every axis y label/.style={at=(current axis.above origin),anchor=south},
    every axis x label/.style={at=(current axis.right of origin),anchor=west},
    height=4cm, width=6cm,
    xtick={6},
    xticklabels={$\sigma$},
    ytick=\empty,
    enlargelimits=false, clip=false, axis on top,
    grid = major
    ]
    \addplot [very thick, magenta!80!black] {gauss(4,1)};
    \addplot [very thick, magenta!50!black] {gauss(5,.5)};
    \addplot [very thick, magenta!20!black] {gauss(5.75,.25)};
    \end{axis}
  \end{tikzpicture}
  \caption{Estimador sesgado consistente}
  \label{fig:estimador_sesgado_consistente}
\end{figure}%

La estimación $\hat\sigma$ es interesante en sí misma porque representa la desviación estándar de $y$ una vez que hemos eliminado los efectos de $x$.
Sin embargo, el principal uso de $\hat\sigma$ es para estimar la desviación estándar de $\hat\beta_1$.
Dado que $\sd(\hat\beta_1) = \sigma / \sqrt{SST_x}$, el estimador natural para esta desviación estándar es
\begin{equation}
\se(\hat\beta_1) = \frac{\hat\sigma}{\sqrt{\sum_{i=1}^n(x_i - \overline x)^2}}.
\end{equation}
A este término se le conoce como el \kw{error estándar} de $\hat\beta_1$.
El error estándar de un estimador nos dá una idea de cuán dispersa es nuestra estimación. Debido a esto, el error estándar juega un rol importante en econometría, y lo usaremos para construir tests estadísticos e intervalos de confianza para distintos métodos que veremos más adelante.

%% Introducción, prefacio:
%<<cap_OLD, child='cap_OLD.Rnw'>>=
%@
 
\nocite{*}
\printbibliography
\printindex


\end{document}
